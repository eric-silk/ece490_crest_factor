\documentclass[conference]{IEEEtran}
\IEEEoverridecommandlockouts
% The preceding line is only needed to identify funding in the first footnote. If that is unneeded, please comment it out.
\usepackage{cite}
\usepackage{amsmath,amssymb,amsfonts}
\usepackage{algorithmic}
\usepackage{graphicx}
\usepackage{textcomp}
\usepackage{xcolor}
\def\BibTeX{{\rm B\kern-.05em{\sc i\kern-.025em b}\kern-.08em
    T\kern-.1667em\lower.7ex\hbox{E}\kern-.125emX}}
\begin{document}

\title{Overview of Crest Factor Minimization Methods\\
{\footnotesize An Extra Paper for ECE490}
}

\author{\IEEEauthorblockN{1\textsuperscript{st} Eric Silk}
    \IEEEauthorblockA{
        \textit{PhD Student, Power And Energy Systems}\\
        \textit{University of Illinois Urbana-Champaign, Electrical Engineering} \\
        Urbana-Champaign, IL\\
        esilk2@illinois.edu}
}

\maketitle

\begin{abstract}
    Crest Factor Minimization is an optimization problem where, given a signal comprised of
    several sinusoids with fixed amplitude and frequency, the relative phases are adjusted
    to minimize the ratio of the peak magnitude to the RMS power over one period. This is
    commonly used to develop optimal test signals for system characterization, among
    other applications. This was one of the last problems I worked on at my prior job
    and wasn't afforded enough time to fully explore the solution space; hence, this paper.
\end{abstract}

\begin{IEEEkeywords}
    crest factor, optimization, L2-norm, Inf-norm, system characterization
\end{IEEEkeywords}

\section{Introduction}
When attempting to characterize systems under test, a common technique is to inject
a test signal with known frequencies and amplitudes and then to measure the output. In
a linear, noiseless system, no special handling would be required; however, real systems
are almost never truly linear, and never truly noiseless. As such, the injected signal
should have maximal power to maximize the signal-to-noise ratio (SNR) while minimizing
the peak amplitude.


\section{Problem Formulation}

\subsection{Ratio of Norms}
The formulation for this problem is fairly straightforward. Given a signal of the form:
\begin{equation}
    f(t,a,\omega,\phi) = \sum_{i=1}^{n}a_i\cos(\omega_it+\phi)
\end{equation}
where $a,\omega\in\mathbb{R}^n$ and are constants, we want to minimze the
peak value of the signal while maximizing the RMS power. This can be
expressed as:
\begin{equation}
    C_f(\phi)
    = \frac{\sup_{t\geq 0} f(t) }{\left(\int_0^\infty f(t,\phi)^2 dt\right)^{\frac{1}{2}}}
    = \frac{\|f(\phi)\|_{\infty}}{\|f(\phi)\|_2}
\end{equation}
The value $C_f$ is known as the the ``crest factor''. The phases $\phi$ that
minimize this can be expressed as:
\begin{equation}
    \phi^* = \arg\min_{\phi} C_f(\phi)
\end{equation}
Of course, it should be plain to see that are an infinite number of solutions,
owing to the periodic nature of the sinusoid; i.e.:
\begin{equation}
    C_{f}(\phi^*) = C_{f}(\phi^*+2k\pi),\ k\in\mathbb{Z}
\end{equation}
as such, one of these phases is set to some constant -- usually, the phase of the
lowest frequency signal set to be zero. Assuming the frequencies are expressed as
a strictly increasing sequence, we then have:
\begin{equation}
    f(t,a,\omega,\phi) = a_i\cos(\omega_1) + \sum_{i=2}^n a_i\cos(\omega_it+\phi_i)
\end{equation}

\subsection{Figures and Tables}
\paragraph{Positioning Figures and Tables} Place figures and tables at the top and
bottom of columns. Avoid placing them in the middle of columns. Large
figures and tables may span across both columns. Figure captions should be
below the figures; table heads should appear above the tables. Insert
figures and tables after they are cited in the text. Use the abbreviation
``Fig.~\ref{fig}'', even at the beginning of a sentence.

\begin{table}[htbp]
    \caption{Table Type Styles}
    \begin{center}
        \begin{tabular}{|c|c|c|c|}
            \hline
            \textbf{Table} & \multicolumn{3}{|c|}{\textbf{Table Column Head}}                                                         \\
            \cline{2-4}
            \textbf{Head}  & \textbf{\textit{Table column subhead}}           & \textbf{\textit{Subhead}} & \textbf{\textit{Subhead}} \\
            \hline
            copy           & More table copy$^{\mathrm{a}}$                   &                           &                           \\
            \hline
            \multicolumn{4}{l}{$^{\mathrm{a}}$Sample of a Table footnote.}
        \end{tabular}
        \label{tab1}
    \end{center}
\end{table}

\begin{figure}[htbp]
    \centerline{\includegraphics{fig1.png}}
    \caption{Example of a figure caption.}
    \label{fig}
\end{figure}

Figure Labels: Use 8 point Times New Roman for Figure labels. Use words
rather than symbols or abbreviations when writing Figure axis labels to
avoid confusing the reader. As an example, write the quantity
``Magnetization'', or ``Magnetization, M'', not just ``M''. If including
units in the label, present them within parentheses. Do not label axes only
with units. In the example, write ``Magnetization (A/m)'' or ``Magnetization
\{A[m(1)]\}'', not just ``A/m''. Do not label axes with a ratio of
quantities and units. For example, write ``Temperature (K)'', not
``Temperature/K''.

\section*{Acknowledgment}

The preferred spelling of the word ``acknowledgment'' in America is without
an ``e'' after the ``g''. Avoid the stilted expression ``one of us (R. B.
G.) thanks $\ldots$''. Instead, try ``R. B. G. thanks$\ldots$''. Put sponsor
acknowledgments in the unnumbered footnote on the first page.

\section*{References}

Please number citations consecutively within brackets \cite{b1}. The
sentence punctuation follows the bracket \cite{b2}. Refer simply to the reference
number, as in \cite{b3}---do not use ``Ref. \cite{b3}'' or ``reference \cite{b3}'' except at
the beginning of a sentence: ``Reference \cite{b3} was the first $\ldots$''

Number footnotes separately in superscripts. Place the actual footnote at
the bottom of the column in which it was cited. Do not put footnotes in the
abstract or reference list. Use letters for table footnotes.

Unless there are six authors or more give all authors' names; do not use
``et al.''. Papers that have not been published, even if they have been
submitted for publication, should be cited as ``unpublished'' \cite{b4}. Papers
that have been accepted for publication should be cited as ``in press'' \cite{b5}.
Capitalize only the first word in a paper title, except for proper nouns and
element symbols.

For papers published in translation journals, please give the English
citation first, followed by the original foreign-language citation \cite{b6}.

\begin{thebibliography}{00}
    \bibitem{b1} G. Eason, B. Noble, and I. N. Sneddon, ``On certain integrals of Lipschitz-Hankel type involving products of Bessel functions,'' Phil. Trans. Roy. Soc. London, vol. A247, pp. 529--551, April 1955.
    \bibitem{b2} J. Clerk Maxwell, A Treatise on Electricity and Magnetism, 3rd ed., vol. 2. Oxford: Clarendon, 1892, pp.68--73.
    \bibitem{b3} I. S. Jacobs and C. P. Bean, ``Fine particles, thin films and exchange anisotropy,'' in Magnetism, vol. III, G. T. Rado and H. Suhl, Eds. New York: Academic, 1963, pp. 271--350.
    \bibitem{b4} K. Elissa, ``Title of paper if known,'' unpublished.
    \bibitem{b5} R. Nicole, ``Title of paper with only first word capitalized,'' J. Name Stand. Abbrev., in press.
    \bibitem{b6} Y. Yorozu, M. Hirano, K. Oka, and Y. Tagawa, ``Electron spectroscopy studies on magneto-optical media and plastic substrate interface,'' IEEE Transl. J. Magn. Japan, vol. 2, pp. 740--741, August 1987 [Digests 9th Annual Conf. Magnetics Japan, p. 301, 1982].
    \bibitem{b7} M. Young, The Technical Writer's Handbook. Mill Valley, CA: University Science, 1989.
\end{thebibliography}
\vspace{12pt}
\color{red}
IEEE conference templates contain guidance text for composing and formatting conference papers. Please ensure that all template text is removed from your conference paper prior to submission to the conference. Failure to remove the template text from your paper may result in your paper not being published.

\end{document}
